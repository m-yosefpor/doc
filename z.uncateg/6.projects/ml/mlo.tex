\documentclass[11pt]{beamer}
%%%%%%%%%%%%%%%%%%%%%%%%%%%%%%%%
\usepackage[utf8]{inputenc}
\usepackage{braket}
\usepackage{amsmath}
%%%%%%%%%%%%%%%%%%%%%%%%%%%%%%%%
\usefonttheme{serif}
%%%%%%%%%%%%%%%%%%%%%%%%%%%%%%%%
\usetheme{Madrid} 
%\usetheme{Copenhagen}
%\usetheme{Berkeley}
%%%%%%%%%%%%%%%%%%%%%%%%%%%%%%%%
%\usecolortheme{beaver}
%\usecolortheme{beetle}
%\usecolortheme{seahorse}
%\usecolortheme{wolverine}

%%%%%%%%%%%%%%%%%%%%%%%%%%%%%%%%
\title[Entanglement Detection]{Quantum Machine Learning}
\subtitle{Entanglement Detection}

 
\author[Yosefpor,Mostaan]
{M.~Yosefpor \and M.~Mostaan}

%\author[Author, Doe]{A.~B.~Arthur\inst{1} \and J.~Doe\inst{2}}



\institute[SUT]{Sharif University of Technology}

\date[Spring 2019]{Spring 2019}

%\logo{\includegraphics[height=1.5cm]{lion-logo.png}}
%%%%%%%%%%%%%%%%%%%%%%%%%%%%%55
\AtBeginSection[]{
\begin{frame}
\frametitle{Table of Contents}
\tableofcontents[currentsection]
\end{frame} 
}

%%%%%%%%%%%%%%%%%%%%%%%%%%%%%%%%%%%%%%%%%%%%%%%%%%%%%%%%%%%%%%%%%%%%%%
\begin{document}
 
\frame{\titlepage}

\begin{frame}
\frametitle{Table of Contents}
\tableofcontents
\end{frame}

%%%%%%%%%%%%%%%%%%%%%%%%%%%%%%%%%%%%%%%%%%%%%%%%%%%%%%%%%%%5
\section{Preliminaries}
%---------------------------------------%
\begin{frame}
\frametitle{State of a system}

\begin{itemize}
 \item  A State of a system is represented by a vector in Hilbert space:
 $$\ket{\psi}$$
 \item  This only happens when we know the exact state of the system. This state is called a pure state.
 \item  Usually in laboratories we don’t have so much information about the system.
\end{itemize}
\end{frame}
%---------------------------------------%
\begin{frame}
\frametitle{State of a system}


\begin{itemize}
 \item  Imagine a ray of photons
 \item  After passing a polaroid we can detect that a portion of photons have a vertical (or horizontal) polarization
 \item  But what about before the polaroid?
 \item  What can we say about the state of photon ray before entering the polaroid?
\end{itemize}

\end{frame}
%---------------------------------------%
\begin{frame}
\frametitle{Density Matrix}

\begin{itemize}
 \item  For representing such systems that we can’t be certain about the exact state of them we use a statistical representation: $$ \braket{A} = \sum_i p_i \braket{\psi_i | A | \psi_i}$$


$i$ : index of all possible states for presumed system

$p_i$  : probability of the system being in $\ket{\psi_i}$

\end{itemize}

\end{frame}


%---------------------------------------%
\begin{frame}
\frametitle{Density Matrix}

\begin{itemize}
 \item  Then we can define : $$\rho = \sum_i p_i \ket{\psi_i}\bra{\psi_i} $$
 \item  so we can write: $$\braket{A} = tr(A\rho) $$
 
\end{itemize}

\end{frame}

%---------------------------------------%
\begin{frame}
\frametitle{Density Matrix}

\begin{itemize}
 \item  It is clear that $\rho$ is a representation of photon rays 
and the system has a \textbf{mixed state}.
 \item  All the information about a system exist in the system's density matrix.

\end{itemize}

\end{frame}

%---------------------------------------%
\begin{frame}
\frametitle{Density Matrices Properties}

\begin{itemize}
 \item  A density matrix representing a physical system should have these following properties:
 \item  $\rho^\dag = \rho$
 \item  $tr(\rho) = 1$
 \item  $\forall a \; \braket{a|\rho|a} \geqslant 0$ or $\rho \geqslant 0$
\end{itemize}

\end{frame}

%---------------------------------------%
\begin{frame}
\frametitle{Pure states}

\begin{itemize}
 \item  Obviously a system is pure when:
    $$i={1}$$
    $$\rho = \ket{\psi}\bra{\psi}$$
 \item  We can separate a pure state from mix:
 \item  $pure \iff \rho^2 = \rho$
 \item  $pure \iff tr(\rho^2)=1 $
\end{itemize}

\end{frame}

%---------------------------------------%
\begin{frame}
\frametitle{Bi-partite systems}

\begin{itemize}
 \item  Imagine a system constructed from 2 parts  ( A , B ) 
 \item  The Hilbert space corresponding to this system is:
    $$\mathcal{H} = \mathcal{H}_A \otimes \mathcal{H}_B $$
 \item  Measuring an observable $M$ on one of the subsystems $M \otimes     \mathbb{I} $:
 \item  $tr_{AB}((M \otimes \mathbb{I})\rho) = tr_A(tr_B((M \otimes \mathbb{I} )\rho)) = tr_A(M \rho_A)$
 \item $\rho_A  = tr_B(\rho)$
\end{itemize}

\end{frame}

%---------------------------------------%
\begin{frame}
\frametitle{Bi-partite systems}

\begin{itemize}
 \item  A vector in corresponding Hilbert space:
    $$\ket{\psi} = \sum_{i,\mu} \psi_{i\mu} \ket{i,\mu}$$
 \item  Or we can write as density matrix:
    $$\rho = \sum_{ij\mu\nu} \rho^{ij}_{\mu\nu} \ket{i}\bra{j} \otimes \ket{\mu}\bra{\nu} $$
\end{itemize}

\end{frame}

%---------------------------------------%
\begin{frame}
\frametitle{Entanglement}

\begin{itemize}
 \item  A density matrix is said to be separable if and only if it can be written as:
 $$\rho_{AB} = \rho_A \otimes \rho_B$$
 
 or:
 
 $$\rho_{AB} = \sum_i p_i \rho_A^{(i)} \otimes \rho_B ^{(i)} \; \wedge    \sum_i p_i =1$$
 \item  otherwise the state is said to be entangled

\end{itemize}

\end{frame}

%---------------------------------------%
\begin{frame}
\frametitle{Qubits}

\begin{itemize}
 \item  A system with two possible states
 \item  The smallest part of a quantum computer
 \item   Two states are often shown in the form: $\ket{0} \, , \, \ket{1}$ 
\end{itemize}

\end{frame}
%---------------------------------------%
\begin{frame}
\frametitle{Peres – Horodecki  criterion}

\begin{itemize}
 \item  Partial transpose : 
 $$ \rho^{T_B} = (\mathbb{I}\otimes T)_{\rho} = \sum_{ij\mu\nu}p^{ij}_{\mu\nu}\ket{i}\bra{j} \otimes (\ket{\mu}\bra{\nu})^T $$
 $$= \sum_{ij\mu\nu} p^{ij}_{\mu\nu} \ket{i}\bra{j} \otimes \ket{\nu}\bra{\mu}$$
\end{itemize}

\end{frame}

%---------------------------------------%
\begin{frame}
\frametitle{Peres – Horodecki  criterion}

\begin{itemize}
 \item  For a separable state:
 $$ \rho_{AB}^T = \sum_i p_i \rho_A^{(i)} \otimes {\rho_B^{T}}^{(i)}$$
 \item  $\rho_B^{(i)}$ is also a density matrix so: $\rho^T_{AB} \geqslant 0$
 \item   so if $\rho_{AB}^T < 0 \implies $ entangled 
 \item  The condition is proved to be sufficient 2*2 and 3*2 Hilbert spaces.
\end{itemize}

\end{frame}

%---------------------------------------%
\begin{frame}
\frametitle{Peres – Horodecki  criterion}

\begin{itemize}
 \item  Recently even a simpler condition for two-qubit systems 
(and only for them) has been pointed out  (Augusiak et al., 2006)
    $$det(\rho_{AB}^T) > 0 \iff separable$$ 

\end{itemize}

\end{frame}

%---------------------------------------%
\begin{frame}
\frametitle{Two qubit system :}

\begin{itemize}
 \item  A two qubit state can be written as below : 
  $$ \rho = \sum_{ij} \Gamma_{ij} \sigma_i \otimes \sigma_j$$
  where $\sigma_i$s are pauli matrices
 \item  So we can determine the exact state of a two qubit system Using 15 measurements
\end{itemize}

\end{frame}

%---------------------------------------%
\begin{frame}
\frametitle{Features}

\begin{itemize}
 \item  With 15 measurement we know all the information about the system
 \item  And we can determine if the given state is entangled or not
 \item  But doing 15 measurement in laboratory is not cost efficient
 \item  So we want to reduce the required features for entanglement detection
\end{itemize}

\end{frame}

%---------------------------------------%
\begin{frame}
\frametitle{Entanglement witness}

\begin{itemize}
 \item Witnesses are necessary conditions to determine a entangled state, but they may not be sufficient conditions
 \item So we can use entanglement witnesses to detect some (but not necessarily all) of the entangled states.
 \item  Entanglement is a resource, because these detected states can be used for multiple applications 
 \item We have used machine learning techniques to detect a considerable portion of entangled states with a high certainty, using just a few measurements
\end{itemize}

\end{frame}






%%%%%%%%%%%%%%%%%%%%%%%%%%%%%%%%%%%%%%%%%%%%%%%%%%%%%%%%%%%%%%%%%
\section{Machine Learning Techniques}

\subsection{Data Generation}
%---------------------------------------%
\begin{frame}
\frametitle{Data Generation}

\begin{itemize}
 \item generating random density matrices using qutip module
 \item supervised learning: labeling
 \item labels: calculating $det(\rho_{AB}^{T_B})$ directly and checking the sign
 
 
\end{itemize}

\end{frame}
%---------------------------------------%
\begin{frame}
\frametitle{Data Generation}

\begin{itemize}
\item Features: results of the measurements on the desity matrices by calculating $tr(\rho_{AB} (\sigma_i \simga_j))$ 
  \[
   Features=
  \left[ {\begin{array}{cccc}
\sigma_x\otimes\sigma_x & \sigma_x\otimes\sigma_y & \sigma_x\otimes\sigma_z & \sigma_x\otimes\ \mathbb{I} 
\\ \sigma_y\otimes\sigma_x & \sigma_y\otimes\sigma_y & \sigma_y\otimes\sigma_z & \sigma_y\otimes\ \mathbb{I}
\\ \sigma_z\otimes\sigma_x & \sigma_z\otimes\sigma_y & \sigma_z\otimes\sigma_z & \sigma_z\otimes\ \mathbb{I}
\\ \mathbb{I}\otimes\sigma_x & \mathbb{I}\otimes\sigma_y & \mathbb{I}\otimes\sigma_z & \mathbb{I}\otimes\ \mathbb{I}
  \end{array} } \right]
\]

\item removing the trivial $1$ from $ \mathbb{I} \otimes \mathbb{I} $

\item reshape to (15 , 1)
\end{itemize}

\end{frame}
%---------------------------------------%
\begin{frame}
\frametitle{Data Generation}

\begin{itemize}
 \item 5,000,000 data generated and labeled in a pandas dataframe
 \item data is on dropbox
 \item size about 600 MiB
\end{itemize}

\end{frame}


\subsection{Data Analysis}
%---------------------------------------%
\begin{frame}
\frametitle{Data Analysis-Histogram}

\includegraphics[width=\textwidth>]{histogram.png}
\end{frame}


%---------------------------------------%
\begin{frame}
\frametitle{Data Analysis-Scatter Matrix}

\includegraphics[width=\textwidth>]{scatter.png}

\end{frame}


%---------------------------------------%
\begin{frame}
\frametitle{Data Analysis-Scatter Matrix}

\includegraphics[width=\textwidth>]{ent_feature.png}

\end{frame}


%---------------------------------------%
\begin{frame}
\frametitle{Data Analysis-Corolation Matrix}
\includegraphics[width=\textwidth>]{corr.png}
\end{frame}

\subsection{Data Preparation}
%---------------------------------------%
\begin{frame}
\frametitle{Data Preparation}

\begin{itemize}
 \item No Missing Data
 \itme sampling : train size
 \item clean up : label encoder
 \item shuffle
 \item standard scaler
 \item polynomial features
 \item train test split
\end{itemize}

\end{frame}

\subsection{Learning and Scoring}
%---------------------------------------%
\begin{frame}
\frametitle{Learning and Scoring}

\begin{itemize}
 \item all 15 features
 \item direct classification : binary labels
 \item classifier from regressor: trained on the entanglement
 \item manually classify : sign of the regressor prediction
 \item with/without polynomial features
\end{itemize}

\end{frame}


%---------------------------------------%
\begin{frame}
\frametitle{Learning and Scoring}


\begin{itemize}
 \item test and train scores using cross validation
 \item test and train time
 \item confusion matrix
 \item  precision, recall, f1 score, AUC
 \item precision-recall curve for each class
 \item learning curve
 \item vallidaion curve
\end{itemize}


\end{frame}



%---------------------------------------%
\begin{frame}
\frametitle{Learning and Scoring}

\includegraphics[width=\textwidth>]{table.png}

\end{frame}

\subsection{Feature Reduction}
%---------------------------------------%
\begin{frame}
\frametitle{Feature Importance}

\begin{itemize}
 \item random forest feature importance
 \item coefs
 \item \includegraphics[width=\textwidth>]{tree.png}
\end{itemize}

\end{frame}


%---------------------------------------%
\begin{frame}
\frametitle{Feature Reduction}

\begin{itemize}
 \item manually top features from feature importance
 \item PCA
 \itme manifold
 \item select from model
 \item RFE : best of all (Random Forest Classifier : 88\% with 6 features)
\end{itemize}

\end{frame}

\subsection{Neural Network}
%---------------------------------------%
\begin{frame}
\frametitle{Neural Network}
 \item \includegraphics[width=\textwidth>]{summary.png}
 \item 99\% test accuracy
\end{frame}


%---------------------------------------%
\begin{frame}
\frametitle{Neural Network}
\includegraphics[width=\textwidth>]{roc.png}

\end{frame}


%---------------------------------------%
\begin{frame}
\frametitle{non-linear auto-encoder}
\includegraphics[width=\textwidth>]{summary_nonlin.png}
\end{frame}


%---------------------------------------%
\begin{frame}
\frametitle{non-linear auto-encoder}
\includegraphics[width=\textwidth>]{roc_nonlin.png}
\end{frame}


%---------------------------------------%
\begin{frame}
\frametitle{linear auto-encoder}
\includegraphics[width=\textwidth>]{summary_lin.png}
\end{frame}


%---------------------------------------%
\begin{frame}
\frametitle{linear auto-encoder}
\includegraphics[width=\textwidth>]{roc_lin.png}
\end{frame}


%---------------------------------------%
\begin{frame}
\frametitle{Other Analysis}

\begin{itemize}
 \item noise robustness of each model to different kinds of noises
 \item features of states which detected correctly by NN 
 \item features of states which falsely detected
\end{itemize}

\end{frame}


%---------------------------------------%
\begin{frame}
\frametitle{Ongoing Research}

\begin{itemize}
 \item higher dimensions bi-partite (qudits)
 \item multi-partite qubits
 \item multi-parite qudits
\end{itemize}

\end{frame}

%---------------------------------------%
\begin{frame}
  \centering \huge
  \emph{Thanks for your attention}
\end{frame}

%%%%%%%%%%%%%%%%%%%%%%%%%%%%%%%%%%%%%%%%%%%%%%%%%%%%%%%%%%%%%%%%%
%%%%%%%%%%%%%%%%%%%%%%%%%%%%%%%%%%%%%%%%%%%%%%%%%%%%%%%%%%%%%%%%%
\end{document}
%%%%%%%%%%%%%%%%%%%%%%%%%%%%%%%%%%%%%%%%%%%%%%%%%%%%%%%%%%%%%%%%%


\begin{comment}

In this slide, some important text will be
\alert{highlighted} because it's important.
Please, don't abuse it.
 
\begin{block}{Remark}
Sample text
\end{block}
 
\begin{alertblock}{Important theorem}
Sample text in red box
\end{alertblock}
 
\begin{examples}
Sample text in green box. The title of the block is ``Examples".
\end{examples}



%---------------------------------------%
\begin{frame}
\frametitle{}

\begin{itemize}
 \item 
 \item  
 \item  
 \item  
\end{itemize}

\end{frame}


\end{comment}
